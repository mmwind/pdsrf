\hypertarget{_detailed_one_sec}{}\section{20.\-20.\-2015}\label{_detailed_one_sec}
Отлажен класс Decision\-Stump +

Добавлен Verbose метод для класса Base\-Classifier +

Добавлен Base\-Classifier.\-get\-Info() +

Decision\-Stump.\-predict(\-Dataset d)

Attribute тип добавлен, но не закончен. Пока остаётся висеть.



 \hypertarget{_detailed_two_sec}{}\section{22.\-20.\-2015}\label{_detailed_two_sec}
Базовый класс генератор сплит-\/кандидатов Base\-Split\-Candidate\-Generator

Абстракция для поддержки различных типов разбиений. Пока остаётся висеть.

Внедрён базовый класс Base для всех объектов

Добавлено базовое исключение для всех объектов проекта

Sample\-Split\-Candidate\-Generator интегрировани в Decision stump

Базовый класс Base выставлен родительским для всех соответствующих классов

Возможность цитирования \cite{louppe2014understanding} добавлена. Для этого требуется latex и perl в P\-A\-T\-H 

 \hypertarget{_detailed_two_sec}{}\section{22.\-20.\-2015}\label{_detailed_two_sec}
Todolist страница добалена 

 \hypertarget{_detailed_three_sec}{}\section{23.\-20.\-2015}\label{_detailed_three_sec}
Добавлен класс Base\-Preprocessor, Randomizer, etc Переработана архитектура проекта 

\hypertarget{_detailed_four_sec}{}\section{26.\-20.\-2015}\label{_detailed_four_sec}
C\-S\-V file support added in draft(debug required)

Pattern class Abstract\-Factory added

Factory pattern used to produce unified interface to file readers.

E\-I\-G\-E\-N error assertion macro redefined to produce exception\hypertarget{_detailed_five_sec}{}\section{27.\-20.\-2015}\label{_detailed_five_sec}
Оказалось, что E\-I\-G\-E\-N error assertion macro выскакивает и в не ошибочных ситуациях. Поэтому пока оставим это без изменений. C\-S\-V Reader закончен\hypertarget{_detailed_six_sec}{}\section{28.\-20.\-2015}\label{_detailed_six_sec}
Добавлены классы Base\-Tree\-Node, Base\-Tree

Исправлен Randomizer из Random\-Split\-Generator

Base\-Split\-Candidate\-Generator.\-get\-Subset\-By\-Split для разделения на подмножества по выбраному сгенерированному сплиту

Теперь сплиты будут работать маскируя сэмплы\hypertarget{_detailed_seven_sec}{}\section{30.\-20.\-2015}\label{_detailed_seven_sec}
Base\-Split\-Candidate\-Generator.\-get\-Subset\-By\-Split перенесён в Base\-Tree.

Добавлена рекурсивная процедура обучения дерева learn\hypertarget{_detailed_seven_sec}{}\section{30.\-20.\-2015}\label{_detailed_seven_sec}
Реализован экспорт дерева в R скрипт для partykit 

 \hypertarget{_detailed_todo_sec}{}\section{T\-O\-D\-Olist}\label{_detailed_todo_sec}
Генеартор сплит-\/кандидатов создаёт массив каждый раз при следующем ветвлении, но это можно делать один раз, а потом просто отсеивать сплиты вне текущей ячейки.



 