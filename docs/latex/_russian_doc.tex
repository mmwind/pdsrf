\section*{Описание программной библиотеки Forest Facotry}

\subsection*{Общая информаци}

Проект имеет расширяемую объектно-\/ориентированную структуру. Каждый компонент может быть модифицирован с минимальными затратами средствами языка C++. Для обеспечения наибольшей гибкости в использовании памяти библиотека исползьует стандарт C++11. Для осуществления векторых и матричных расчётов используется библиотека Eigen версии 3.

Все средства библиотеки находятся в пространстве имён ffactory. Все объекты не являющиеся контейнерами являются потомками класса Base.

\subsection*{Контейнеры}


\begin{DoxyItemize}
\item Sample
\item Dataset
\item Data\-Ranges
\item Attribute
\item Prediction
\end{DoxyItemize}

\subsection*{Абстрактный классификатор}

Любой классификатор данной библиотеки наследуется от класса Base\-Classifier. Этот класс приводит унифицированный интерфейс в который входят следующие методы\-:


\begin{DoxyItemize}
\item train
\item predict
\item test
\end{DoxyItemize}

Стандартный код использования классификатора может быть такой\-: \begin{DoxyVerb}    # load dataset from file
        Dataset d;
        CsvFileReader r;
        r.setDataset(&d);
        r.setFilename("test.csv");
        r.setDelimiter(';');
        r.read();
    # train and test classifier
    # BaseTree is derived from BaseClassifier)
        BaseTree tree;
        tree.train(&d);
        std::cout <<"Train accuracy: "<< tree.test(&d) << etd::endl;
\end{DoxyVerb}


\subsubsection*{Attribute}

Три типа атрибутов
\begin{DoxyItemize}
\item A\-T\-T\-R\-\_\-\-C\-O\-N\-T\-I\-N\-U\-O\-U\-S,
\item A\-T\-T\-R\-\_\-\-I\-N\-T\-E\-G\-E\-R,
\item A\-T\-T\-R\-\_\-\-C\-A\-T\-E\-G\-O\-R\-I\-A\-L
\end{DoxyItemize}

которые означают, соответственно, непрерывные, целые или категориальные переменные.

\subsubsection*{Data\-Ranges}

Используется для работы с границами разбиений. Класс используется в Partitioin\-Statistics. В случае непрерывной переменной хранятся границы в которых находятся целевые точки. Если же переменная категориальная, то если нужно указать значение, то верхняя и нижняя граница выставляется в требуемое значение.

\subsubsection*{Dataset}

Хранит в себе весь набор данных и его описание. Описание каждого из признаков и целевой переменной хранится в виде массива Attibute. При использовании отдельно, без считывания набора данных из файла, требуется задать размерности признакового пространства ({\bfseries set\-Num\-Features}) и количество классов ({\bfseries set\-Num\-Classes}), затем инициализировать внутренние структуры класса c с помощью Init. После этого можно добавлять точки.

\subsubsection*{Partition\-Statistics}

Вспомогательный класс, который содержит информацию о точках в заданной области пространства признаков. Такую как их количество, распределение по классам.

\subsection*{Операции с данными}

Пакет содержит унифицированные классы для работы с данными. Частично поддерживаются популярные форматы данных, такие как C\-S\-V и A\-R\-F\-F.

\subsection*{Классификаторы}

Абстрактный класс Base\-Classifier

\subsubsection*{Ансамбли классификаторов}

Абстрактный класс Base\-Ensemble\-Classifier

\subsubsection*{Деревья решений}

Все классификаторы принадлежат к абстрактному классу Base\-Tree. В настоящий момент поддерживаются только аксиально-\/параллельные разбиения.

\paragraph*{Генератор кандидатов на разбиение}

Абстрактный класс {\bfseries Base\-Split\-Candidate\-Generator}

\paragraph*{Измеритель качества разбиения}

Абстрактный класс {\bfseries Base\-Split\-Quality\-Measurer}

\paragraph*{Binary\-Split}

Этот класс используется для разбиении пространства признаков, он содержит в себе номер атрибута(признака) по которому происходит разбиение, его тип, значение и также качество.

\subsection*{Соглашения}

\begin{DoxyVerb}Любой класс имеет декларации для умных указателей на него. Для этого достаточно дописать соответствующий префикс.
Например, DataVectorUniquePtr это std::unique_ptr < DataVector >.
Для этого используется макрос DEFINE_PTR(CLASSNAME).\end{DoxyVerb}
 